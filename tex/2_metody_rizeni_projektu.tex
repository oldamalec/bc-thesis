\chapter{Metody řízení projektu} \label{methods}

V druhé kapitole bych se chtěl věnovat především tématu metodik vývoje softwaru. Cílem je krátce čtenáře seznámit se standardními a agilními metodami vývoje softwaru a zaměřit se na možnosti jejich aplikace pro vývoj DBS portálu, který byl popisován v kapitole \ref{DBSportal}



\section{Tradiční metody} \label{methods:traditional}

\subsection{Vodopádový model} \label{methods:waterfall}

Jak zmiňuje \cite{kadlec}, nejedná se přímo o metodiku, ale pouze o životní cyklus. Je charakteristický tím, že všechny fáze vývoje jsou prováděny postupně za sebou a není možné se vracet. Případné změny je možní zpracovat až v rámci údržby, která vývoj vždy vrátí do jedné z předchozích etap a následně musí proběhnout i etapy následující.

\begin{itemize}
	\item Definice problému
	\item Specifikace požadavků
	\item Návrh
	\item Implementace
	\item Integrace, testování
	\item Údržba
\end{itemize}

Tento model vznikl již v roce 1970 a pro vývoj softwaru, který lze typicky nasadit ještě před jeho kompletním dokončením a poté iterativně dále rozvíjet, je nevhodný.

\subsection{Spirálový model} \label{methods:spiral}

Vzniku Spirálového modelu dala za vznik především kritika Vodopádového modelu. Hlavní novinkou zde byl \emph{iterativní přístup} a opakovaná \emph{analýza rizik}. V tomto modelu se stále opakují fáze \emph{Stanovení cílů}, \emph{Analýza rizik}, \emph{Vývoj a testování} a \emph{Plánování}. Při každém průchodu těmito fázemi se provádí odlišná činnost než v předchozí iteraci (kromě analýzy rizik, ta je prováděna vždy stejně), například fáze vývoje v první iteraci pracuje pouze s koncepty, v příští iteraci specifikuje požadavky, v další navrhuje architekturu a teprve poté implementuje a testuje.\\
Stále se však jedná převážně o jednosměrnou cestu, která nejen na počátku ale i v průběhu obsahuje velké množství analýz a návrhů. Pro potřeby vývoje DBS portálu, kde se každý rok mění složení studentů, kteří projekt realizují, je tedy nevhodný.

TODO more std methods?



\section{Agilní metody} \label{methods:agile}

Agilní metody řízení vznikly především z důvodu neustále se měnících požadavků na software a rychlejší reakci na požadované změny. Tradiční metody nebyly schopné reagovat na změny požadavků dostatečně rychle. Především když se začneme bavit o vývoji softwaru, kde se neustále mění dostupné technologie a požadavky na finální verzi, jsou tradiční metody zcela nevhodným modelem vývoje.

Rozdíl mezi tradiční a agilní metodikou lze popsat také jejich přístupem k \emph{Funkcionalitě}, \emph{Času} a \emph{Zdrojům} \cite{kadlec}. Zatímco u tradičních metod je funkcionalita stanovena na začátku a musí být dosažena pomocí \emph{nějakého} času a \emph{nějakého} množství zdrojů, u agilních metod je to přesně naopak. Typicky je stanoven termín a dostupné zdroje a funkcionalita \emph{nějaká} bude.

Připomeňme si také \emph{Manifesto for Agile Software Development} \cite{manifesto}, který vznikl v roce 2001:

\begin{quote}
Objevujeme lepší způsoby vývoje software tím,
že jej tvoříme a pomáháme při jeho tvorbě ostatním.
Při této práci jsme dospěli k těmto hodnotám:
\begin{itemize}
	\item \emph{Jednotlivci a interakce} před procesy a nástroji
	\item \emph{Fungující software} před vyčerpávající dokumentací
	\item \emph{Spolupráce se zákazníkem} před vyjednáváním o smlouvě
	\item \emph{Reagování na změny} před dodržováním plánu
\end{itemize}
Jakkoliv jsou body napravo hodnotné,
bodů nalevo si ceníme více.

\begin{multicols}{3}
Kent Beck\\
Mike Beedle\\
Arie van Bennekum\\
Alistair Cockburn\\
Ward Cunningham\\
Martin Fowler\\
James Grenning\\
Jim Highsmith\\
Andrew Hunt\\
Ron Jeffries\\
Jon Kern\\
Brian Marick\\
Robert C. Martin\\
Steve Mellor\\
Ken Schwaber\\
Jeff Sutherland\\
Dave Thomas\\
\end{multicols}

{\color{gray}
© 2001, výše zmínění autoři
Toto prohlášení může být volně kopírováno v jakékoli formě,
ale pouze v plném rozsahu včetně této poznámky.
}

\end{quote}

Je tedy zřejmé, že Agilní metodiky více pracují s vývojáři samotnými, a umožňuji i potlačení některých postupů pro dobro celku a dovolují "zanedbat" i náležitosti jako například dokumentaci. Následně se na některé agilní metodiky zaměříme více:

\subsection{Extrémní programování} \label{methods:XP} \footnote{Někteří autoři (Wysocki \cite{wysocki}) oddělují XP od agilních metod, pravděpodobně proto, že vzniklo o 2 roky dříve než \emph{Agile manifesto} \cite{manifesto} z roku 2001. Jelikož však XP postavilo základ pro agilní metodiky, zahrnu jej v tomto textu, stejně jako například Kadlec \cite{kadlec} do této kapitoly.}

XP rozvíjí standardní postupy, avšak posouvá je až do extrémů:
\begin{itemize}
	\item Vždy budeme pracovat na co nejjednodušší verzi, která splňuje požadavky
	\item Pokud se osvědčuje \emph{revize kódu}, budeme neustále revidovat kód \ldots
	\item Pokud se odvědčil \emph{návrh}, budeme neustále navrhovat a vylepšovat specifikaci \ldots
	\item Pokud se osvědčilo \emph{testování}, budeme neustále testovat \ldots
	\item Narozdíl od tradičních metodik, které najednou dodají velký balík nových funkcionalit, bude XP dodávat i ty nejmenší funkční kousky. Toto vede na \emph{krátké iterace}.
\end{itemize}

Tyto postupy přivedené do extrémů jsou reprezentovány v \emph{12 základních postupech XP}, se kterými se seznámíme níže.\\
Ještě než se dostaneme k samotným postupům, potřebujeme se seznámit s pojmem \emph{karta zadání.}

\paragraph{Karty zadání a karty úkolů} \label{methods:XP:issues}
Karty zadání vznikají během plánování návrhu softwaru. Tohoto plánování se účastní všichni vývojáři, vedení a i zákazník. Typicky jedna karta popisuje jednu funkcionalitu či jeden požadavek. Kartě náleží vlastnosti jako \emph{datum vytvoření}, \emph{stav}, \emph{priorita}, \emph{odhad vyřešení} a také samozřejmě vlastní zadání úkolu a dodatečné poznámky. Když se poté na jedné kartě pracuje, je aktualizována průběžnými poznámkami, které popisují které části již byly splněny, co je potřeba udělat atp.\\
Kromě karet zadání existují v XP také \emph{karty úkolů}. Tyto vznikají nikoliv na začátku celého procesu vývoje, ale na začátku jednotlivé iterace.

\paragraph{12 základních postupů XP}
\begin{enumerate}
	\item \emph{Plánovací hra:} Probíhá jednak na začátku vývoje v rámci návrhu celé aplikace, tak na začátku jednotlivých iterací. Jejím výstupem jsou \emph{Karty zadání} a \emph{Karty úkolů}, popisované v předchozím odstavci.
	\item \emph{Malé verze:} Verze jsou dodávány neustále. I nejmenší změna funkcionality, která přináší užitek pro zákazníka je vydána jako nová verze.
	\item \emph{Metafora:} Celý systém je popisován jednoduchým příběhem, jak má ve výsledku fungovat. Tento příběh je sdílen jak vývojáři a managementem, tak zákazníkem a případně dalšími zainteresovanými osobami. Tato metafora u XP nahrazuje tradiční popis architektury.
	\item \emph{Jednoduchý návrh:} XP navrhuje vždy pouze to, co je přímo požadováno. Žádné přidáné vlastnosti nechce do návrhu zahrnovat, snaží se udělat minimum pro to, aby byly splněny požadavky.
	\item \emph{Testování:} Během vývoje jsou neustále psány i testy pro nově vznikající komponenty. Nikdy nesmí existovat jednotka, která nemá vlastní unit test. Testy dokonce píší i sami zákazníci, kteří tvoří testy funkcionality.
	\item \emph{Refaktorizace:} Refaktorizace je změna zdrojového kódu, která ovšem nemění jeho chování. Je prováděna z důvodu zvýšení čitelnosti kódu pro ostatní programátory či jeho zefektivnění. V XP by měl programátor vždy buď pouze implementovat novou funkčnost, nebo refaktorizovat stávající kód.
	\item \emph{Párové programování:} Zdrojový kód je vždy psán dvěma programátory u jednoho počítače. Zatímco jeden píše kód, druhý sleduje postup a přemýšlí nad dalšími souvislostmi: Je tato funkce napsána správně? Zapadá do celkového konceptu? Změní tato úprava chování jiného modulu? Je potřeba napsat nový test? \ldots
	\item \emph{Společné vlastnictví:} Zatímco některé jiné metodiky přiřazují jednotlivé třídy určitým programátorům, které za ně poté mají zodpovědnost, v XP je to právě naopak. Všichni mají zodpovědnost za celý kód a kdokoliv tak může měnit libovolnou součást systému.
	\item \emph{Nepřetržitá integrace:} Systém je sestaven a otestván vždy, jakmile je to možné, typicky tedy ihned po dokončení nové funkcionality.
	\item \emph{Čtyčicetihodinový pracovní týden:} XP klade důraz i na spokojenost lidských zdrojů. Snaží se tedy o nepřetěžování programátorů, jelikož při jejich únavě se razantně snižuje kvalita jejich výstupu.
	\item \emph{Zákazník na pracovišti:} Součastí týmu by měl být i sám zákazník, neboli expert na cílovou doménu. Jeho účelem je odpovídat na dotazy, tvořit testy či určovat priority.
	\item \emph{Standardy pro psaní zdrojového kódu:} Jelikož XP nedefinuje žádné standardní dokumenty, jsou veškeré informaci o projektu obsaženy ve zdojových kódech samotných. Proto je potřeba, aby byly rozumně strukturované, čitelné a pochopitelné.
\end{enumerate}

Některé z těchto postupů jsou jistým způsobem používány i při vývoji DBS portálu, více v sekci \ref{methods:dbs}.

\subsection{Scrum} \label{methods:scrum}

Scrum je charakterizován především pojmy \emph{Sprint} a \emph{Scrum meeting}. První z nich má trvání většinou jeden měsíc a opakuje se několikrát. Jeho cílem je přinést do výsledného produktu nové funkcionality. Naopak Scrum meeting je \emph{každodenní} schůzka, na které se shrnou nově dokončené úkoly a stanoví se, na čem se bude dále pracovat. Schůzku vede tzv. \emph{Scrum master}. Typicky probíhá ve stoje a netrvá déle než 30 minut. Dalo by se říci, že tyto každodenní schůzky jsou stěžejní aktivitou metodiky Scrum. Od těchto schůzek se také odvíjí velikost týmu: Scrum preferuje malé týmy mezi třemi až šesti členy.

Pro potřeby DBS projektu je tedy Scrum také nevhodný, především jelikož předpokládá každodenní schůzky.

\subsection{Lean development} \label{methods:lean}

TODO ?

\subsection{Feature driven development} \label{methods:fdd}

Jak název napovídá, FDD se soustředí především na \emph{featury}, neboli jednotlivé vlastnosti, funkce či rysy výsledného softwaru. Na počátku je vytvořen základní model softwaru, který je dále členěn na jednotlivé vlastnosti. Vývoj probíhá zpravidla v dvoutýdenních iteracích, jejichž výsledkem by měl vždy být fungující software, který je možné představit zákazníkovi.

Klíčový pojem \emph{feature (vlastnost)}  lze popsat jako samostatnou část systému, která splňuje následující parametry:
\begin{itemize}
	\item \emph{měritelnost}: vlastnost je jasně uchopitelná komponenta systému, kterou lze porovnat s požadavkem zákazníka
	\item \emph{srozumitelnost}: programátor musí rozumět tomu, jak má vlastnost fungovat a co je jejím cílem
	\item \emph{realizovatelnost}: programátor musí vědět, zda je schopen vlastnost realizovat v odpovídajícím časovém úseky (typicky iterace 2 týdny). V opačném případě je většinou potřeba vlastnost rozdělit na více menších, a na těch pracovat samostatně.
\end{itemize}

V tuto chvíli by se mohlo zdát, že FDD vůbec nedisponuje \emph{návrhem a modelováním}. Ve skutečnosti je tomu právě naopak, společně s FDD je často zmiňován ještě \emph{model-driven approach}. Jelikož vlastnosti jsou klíčovou stavební jednotkou FDD, je potřeba je kvalitně navrhnout a namodelovat.

Posloupnost kroků při vývoji pomocí FDD zachycuje následující seznam:
\begin{enumerate}
	\item vytvoření celkového (globálního) modelu
	\item vypracování podrobného seznamu vlastností
	\item plánování podle vlastností
	\item návrh podle vlastností
	\item implementace
\end{enumerate}
Dvě poslední se poté stále opakují, dokud existují další vlastnosti, které je potřeba pro softwaru přidávat.

Vzhledem k vývoji DBS portálu zde stojí za zmínku především fáze návrhu a implementace:

\paragraph{Návrh}
Hlavní programátor v této fázi vybere vlastnosti, které budou realizovány v následující iteraci. Vlastnosti volí podle jejich priority, vzájemných návazností a dlaších kritérií. Poté jsou stanoveny třídy, kterých se vlastnost týká a je předána programátorům, kteří za danou třídu zodpovídají.\\
Jelikož různé vlastnosti se vždy týkají různých tříd, jsou programátoři děleni do týmů dle potřeby a jednotlivé týmy se tak pro každou iteraci obměňují.\\
Tento nově sestavený tým má poté za úkol připravit návrh implementace vlasnosti, na jehož základě je poté naprogramována.

\paragraph{Implementace}
Týmy sestavené v předchozí fázi pracují na samotné funkcionalitě nové vlastnosti. Společně s programovým kódem jsou tvořeny i testy, a to především \emph{unit testy}.\\
Jakmile je vlastnost dokončena, je vložena do sdíleného prostoru, u FDD nazývaného \emph{class repository}\\
Hlavní programátor poté dokončené vlastnosti integruje do předchozí verze hlavní aplikace.

Typicky na projektu pracuje několik týmů a hlavní progragramátor poté pouze volí nové vlastnosti k implementaci a integruje hotové vlastnosti do výsledné aplikace.

Výše popsaný průběh práce hrubě odpovídá i organizaci vývoje DBS projektu, které je popsáno v sekci \ref{methods:dbs}.

TODO popsat týmy podle vlastností? (Kadlec 194)

\subsection{Test driven development} \label{methods:tdd}

TODO



\section{Zvolené řešení pro DBS projekt} \label{methods:dbs}

Jak bylo zmíněno v sekcích o \emph{\nameref{methods:XP}} a \emph{\nameref{methods:fdd}}, DBS projekt je vyvíjem pomocí metody na pomezí těchto dvou metodik.
TODO
