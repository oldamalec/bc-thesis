\chapter{Metody řízení projektu} \label{methods}

V druhé kapitole bych se chtěl věnovat především tématu metodik vývoje softwaru. Cílem je krátce čtenáře seznámit se standardními a agilními metodami vývoje softwaru a zaměřit se na možnosti jejich aplikace pro vývoj DBS portálu, který byl popisován v kapitole \ref{DBSportal}

\section{Tradiční metody} \label{methods:traditional}

\subsection{Vodopádový model} \label{methods:waterfall}

Jak zmiňuje \cite{kadlec}, nejedná se přímo o metodiku, ale pouze o životní cyklus. Je charakteristický tím, že všechny fáze vývoje jsou prováděny postupně za sebou a není možné se vracet. Případné změny je možní zpracovat až v rámci údržby, která vývoj vždy vrátí do jedné z předchozích etap a následně musí proběhnout i etapy následující.
Tento model vznikl již v roce 1970 a dnes se tedy dá považovat za zastaralý. Pro vývoj DBS portálu je tedy zcela jistě nevhodný.

\subsection{Spirálový model} \label{methods:spiral}

Vzniku Spirálového modelu dala za vznik především kritika Vodopádového modelu. Hlavní novinkou zde byl \emph{iterativní přístup} a opakovaná \emph{analýza rizik}. V tomto modelu se stále opakují fáze \emph{Stanovení cílů}, \emph{Analýza rizik}, \emph{Vývoj a testování} a \emph{Plánování}. Při každém průchodu těmito fázemi se provádí odlišná činnost než v předchozí iteraci (kromě analýzy rizik, ta je prováděna vždy stejně), například fáze vývoje v první iteraci pracuje pouze s koncepty, v příští iteraci specifikuje požadavky, v další navrhuje architekturu a teprve poté implementuje a testuje.\\
Stále se však jedná převážně o jednosměrnou cestu, která nejen na počátku ale i v průběhu obsahuje velké množství analýz a návrhů. Pro potřeby vývoje DBS portálu, kde se každý rok mění složení studentů, je tedy nevhodný.

TODO more std methods?

\section{Agilní metody} \label{methods:agile}

Agilní metody řízení vznikly především z důvodu neustále se měnících požadavků na software a rychlejší reakci na požadované změny. Tradiční metody nebyly schopné reagovat na změny požadavků dostatečně rychle. Především když se začneme bavit o vývoji softwaru, kde se neustále mění dostupné technologie a požadavky na finální verzi, jsou tradiční metody zcela nevhodným modelem vývoje.

Rozdíl mezi tradiční a agilní metodikou lze popsat také jejich přístupem k \emph{Funkcionalitě}, \emph{Času} a \emph{Zdrojům} \cite{kadlec}. Zatímco u tradičních metod je funkcionalita stanovena na začátku a musí být dosažena pomocí \emph{nějakého} času a \emph{nějakého} množství zdrojů, u agilních metod je to přesně naopak. Typicky je stanoven termín a dostupné zdroje a funkcionalita \emph{nějaká} bude.

Připomeňme si také \emph{Manifesto for Agile Software Development}\cite{manifesto}, který vznikl v roce 2001:

\begin{quote}
Objevujeme lepší způsoby vývoje software tím,
že jej tvoříme a pomáháme při jeho tvorbě ostatním.
Při této práci jsme dospěli k těmto hodnotám:
\begin{itemize}
	\item \emph{Jednotlivci a interakce} před procesy a nástroji
	\item \emph{Fungující software} před vyčerpávající dokumentací
	\item \emph{Spolupráce se zákazníkem} před vyjednáváním o smlouvě
	\item \emph{Reagování na změny} před dodržováním plánu
\end{itemize}
Jakkoliv jsou body napravo hodnotné,
bodů nalevo si ceníme více.

© 2001, výše zmínění autoři
Toto prohlášení může být volně kopírováno v jakékoli formě,
ale pouze v plném rozsahu včetně této poznámky. 

TODO change the last par - smaller font, grey color...

\end{quote}

Je tedy zřejmé, že Agilní metodiky více pracují s vývojáři samotnými, a umožňuji i potlačení některých postupů pro dobro celkua dovolují "zanedbat" i náležitosti jako třeba dokumentaci. Následně se na některé agilní metodiky zaměříme více:

\subsection{Extrémní programování}

XP rozvíjí standardní postupy, avšak posouvá je až do extrémů.
\begin{itemize}
	\item Vždy budeme pracovat na co nejjednodušší verzi, která splňuje požadavky
	\item Pokud se osvědčuje \emph{revize kódu}, budeme neustále revidovat kód \ldots
	\item Pokus se odvědčil \emph{návrh}, budeme neustále navrhovat a vylepšovat specifikaci \ldots
	\item Pokud se osvědčilo \emph{testování}, budeme neustále testovat \ldots
\end{itemize}
TODO rozvést, proč ne pro DBS...

\subsection{Scrum}
