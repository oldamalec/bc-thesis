\newglossaryentry{gCI}{
	name=Continuous integration,
	description={
		Systém automatického sestavováno a testování změn zdrojového kódu
		}
	}
\newglossaryentry{gDML}{
	name=Data manipulation language,
	description={
		Soubor klíčových slov se stanovenou syntaxí, podobný programovacímu jazyku, určený pro manipulaci s daty uloženými v databázi. Nejznáměnší DML je součástí SQL
		}
	}
\newglossaryentry{gGrido}{
	name=Grido,
	description={
		PHP komponenta umožňující jednoduchý výpis dat do tabulky, jejich řazení, filtrování, stránkování a hromadné akce
		}
	}
\newglossaryentry{gNette}{
	name=Nette,
	description={
		Webový framework pro PHP
		}
	}
\newglossaryentry{gOBS}{
	name=Organization Breakdown Structure,
	description={
		Hierarchický model popisující rozdělení odpovědnosti za jednotlivé části projektu
		}
	}
\newglossaryentry{gSQLite}{
	name=SQLite,
	description={
		Jednoduchý databázový stroj, který uchovává celou databázi pouze v jednom souboru
		}
	}
\newglossaryentry{gcommit}{
	name=commit,
	description={
		Aplikování či přijmutí nových změn zdrojového kódu. V tomto textu používáno především v souvislosti s \emph{Gitem}
		}
	}
\newglossaryentry{gfeature}{
	name=Feature,
	description={
		Vlastnost, funkce, rys.
		}
	}
\newglossaryentry{gframework}{
	name=Framework,
	description={
		Softwarová struktura, která slouží jako podpora pro pohodlnější programování. Může obsahovat podpůrné funkce, knihovny či nástroje pro efektivnější, bezpečnější a pohodlnější vývoj softwaru
		}
	}
\newglossaryentry{ggitlab}{
	name=Gitlab,
	description={
		Nástavba nad Gitem ulehčující používání repozitáře ve webovém rozhraní
		}
	}
\newglossaryentry{ggit}{
	name=Git,
	description={
		Distribuovaný systém pro správu verzí vhodný především pro soubory s textovým obsahem
		}
	}
\newglossaryentry{gmvp}{
	name=Model-View-Presenter,
	description={
		\emph{Návrhový vzor} softwaru, který klade důraz na oddělení jednotlivých vrstev aplikace. Využívá jej například Nette Framework.
		}
	}
\newglossaryentry{gpush}{
	name=push,
	description={
		Odeslání \emph{commitu} do sdíleného repozitáře, například Gitlabu
		}
	}
\newglossaryentry{gslack}{
	name=Slack,
	description={
		Komunikační nástroj pro týmy
		}
	}
\newglossaryentry{gcron}{
	name=TODO?,
	description={
		TODO-TODO
		}
	}