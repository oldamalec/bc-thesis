Téma \emph{projektového řízení} je potřeba řešit u každého alespoň středně velkého projektu. Portál pro podporu výuky předmětu BI-DBS se již do takových rozměrů také dostal, zvlášť když je zároveň potřeba udržovat běh nasazené aplikace, opravovat chyby, které objevují koncoví uživatelé, a řídit další vývoj nových funkcí. Kromě řízení projektu obsahuje také práce téma \emph{infrastruktury} vývoje softwaru, která zajišťuje efektivnější vývoj, komunikaci a plánování. Má působnost při řízení tohoto projektu a jeho infrastruktury by měla přispět k vyšší kvalitě používání zmíněné aplikace. Z výstupu by tedy měli těžit všichni studenti a vyučující předmětu BI-DBS.\\
Má motivace k řešení tohoto tématu vznikla ve chvíli, kdy jsem se v předmětu BI-SP2 stal vedoucím jednoho z týmů, které aplikaci vyvíjely pod vedením Jiřího Hunky. Viděl jsem, že mé programátorské schopnosti nedosahují takových úrovní, jako některých ostatních studentů, kteří například naprogramovali celou kostru nového systému, zato jsem však měl cit pro procesy v projektovém řízení, svědomitost při záznamech stráveného času nebo zájem o nastavování postupů vývoje a konfigurace nových podpůrných služeb, potřebných pro běh i vývoj aplikace.\\
V práci nejprve seznámím čtenáře s předmětem BI-DBS, pro který je portál vyvíjen a přiblížím podobu řízení projektu, jak probíhalo před započetím mé práce. Následně se zabývám analýzou metodik vývoje softwaru, ze kterých tvořím závěry pro jejich využití při vývoji DBS portálu. Třetí kapitola obsahuje jak teorii, tak praxi při řízení lidských zdrojů, což je jedním z mých hlavních zaměření při práci na projektu. Čtvrtá a poslední kapitola popisuje veškeré podpůrné nástroje, ať už se jedná o konfigurace serveru nebo služby používané realizačním týmem, k jejichž nastavení nebo definici použivání jsem jistou měrou přispěl.\\
Práce by se dala zařadit k ostatním bakalářským a diplomovým pracem, které vznikly pro účely DBS projektu. Jejich autoři, chronologicky seřazeni podle termínu dokončení práce, jsou: Jan Sýkora, Martin Kubiš, Filip Glazar, Petr Pejša, Jiří Slavotínek, Tomáš Fedor, Pavel Kovář a Milan Vlasák.
