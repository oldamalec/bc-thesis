Práce se zaměřuje na efektivní řízení softwarového projektu, jehož cílem je vyvinout podpůrnou webovou aplikaci pro výuku předmětu BI-DBS na FIT ČVUT. Práce popisuje problémy s vedením vícečlenných týmů, zabývá se problematikou nasazování a udržitelnosti běhu výsledného softwaru a jeho průběžného vývoje. Cílem práce bylo navrhnout a začít používat postupy a nástroje, díky kterým se efektivně skloubí běžící systém a jeho průběžný vývoj za minimálního výskytu chyb. Jako metodika vývoje byla zvolena kombinace Extrémního programování a Feature-driven development, průběh práce je sledován v Redmine. Zdrojový kód aplikace je verzován a testován na Gitlabu. Poslední kapitola také obsahuje praktické ukázky konfigurace některých služeb potřebných pro běh samotné aplikace. Z práce lze vycházet při řízení jiných projektů obdobných rozměrů či pokračování v řízení stejného projektu.