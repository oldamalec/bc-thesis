Práce určitě splnila své předpoklady, projektové řízení celého projektu se posunulo na mnohem vyšší úroveň. Podařilo se mi nastavit postup práce v Redmine tak, aby byly jasně stanovené role vedoucích týmů a byl kladen důraz na jejich důležitost při kontrole výstupu, zanalyzoval jsem i psychologické jevy nastávající při práci s lidmi a pokusil jsem se jejich nežádoucí efekty co nejvíce eliminovat.\\
Studenti realizující portál v rámci předmětů BI-SP1 a BI-SP2 mají mnohem kvalitnější feedback a i nově zavedené nástroje zvyšují efektivitu práce. Informovanost napříč všemi službami využívaných v projektu se značne zlepšila díky provázání jednotlivých služeb, ať už přes nabízená API nebo pomocí vlastních úprav zdrojových kódu.\\
Co se infrastruktury týká, některá nasazná řešení byla \emph{nutná} pro umožnění používání aplikace cílovým uživatelům, jiná zlepšují efektivitu vývoje. I přes množství nově nasazených služeb je u infrastruktury stále možnost ji dále vylepšovat, jednak novými nástroji, jednak zlepšením konfigurací těch současných. Tyto možnosti jsem nastínil na konci kapitoly, která se infrastrukturou zabývá.\\
Během práce jsem také řešil řadu akutních chyb, které se i přes veškerou naši snahu dostaly do produkční verze. Při prvotním nasazování aplikace do výuky jsem se osobně účastnil testů a zkoušek a poskytoval podporu jak učitelům, tak studentům. Při jednom testu nastala i situace, kdy jsem \emph{během běžících testů} připravil opravu kritické chyby a nasadil ji na server, zatímco studenti pracovali na zadaných otázkách.\\
Při pohledu na čas, který jsem u projektu strávil, lze započítat nejprve období, kdy jsem byl studentem BI-SP1 a BI-SP2. Za tyto dva semestry jsem strávený čas přesně zaznamenával, a jednalo se o 362 hodin. Následně jsem se přesunul do role projektového manažera, kde jsem již hodiny nezaznamenával, dle mých odhadů se však každý semestr jedná o zhruba 150 - 200 hodin. V této roli jsem již třetím semestrem, výsledný strávený čas se tedy blíží 1000 hodinám.
