Práce určitě splnila své předpoklady, projektové řízení celého projektu se posunulo na mnohem vyšší úroveň. Podařilo se mi nastavit postup práce v Redmine tak, aby byly jasně stanovené role vedoucích týmů a byl kladen důraz na jejich důležitost při kontrole výstupu, zanalyzoval jsem i psychologické jevy nastávající při práci s lidmi a pokusil jsem se jejich nežádoucí efekty co nejvíce eliminovat.\\
Studenti realizující portál v rámci předmětů BI-SP1 a BI-SP2 mají mnohem kvalitnější feedback a i nově zadené nástroje zvyšují efektivitu práce. Informovanost napříč všemi službami využívaných v projektu se značne zlepšila díky provázání jednotlivých služeb, ať už přes nabízená API nebo pomocí vlastních úprav zdrojových kódu.\\
Co se infrastruktury týká, některá nasazná řešení byla \emph{nutná} pro umožnění používání aplikace cílovým uživatelům, jiná zlepšují efektivitu vývoje. I přes množství nově nasazených služeb je u infrastruktury stále možnost ji dále vylepšovat, jednak novými nástroji, jednak zlepšením konfigurací těch současných. Tyto možnost jsem nastínil na konci kapitoly, která se infrastrukturou zabývá.