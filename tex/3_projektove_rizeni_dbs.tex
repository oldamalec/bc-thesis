\chapter{Řízení DBS projektu} \label{DBSmanagement}

Kapitola popisuje řízení DBS projektu, jeho předešlý stav, návrhy na vylepšení a zhodnocení aplikovaných řešení. Zabývá se především efektivností řízení lidských zdrojů, komunikace a použitých technologií.

\section{Řízení lidských zdrojů}

Projekt je vývíjen již od roku 2014 v rámci předmětů BI-SP1 a navazujícího BI-SP2. Někteří studenti po dokončení BI-SP2 pokračují v práci na systému v rámci své bakalářské práce. Výjimečně jsou také vypsány samostatné bakalářské či diplomové práce zaměřující se na specifickou komponentu systému.
Ze studentů předmětů BI-SP1, případně BI-SP2 jsou typycky sestaveny 2-3 týmy po 4-5 členech.
TODO vedoucí týmu atp.

\section{Řízení komunikace}

Hlavním stěžejním bodem komunikace byla vždy pravidelná týdenní schůzka, na které se sešly řešitelské týmy s vedoucím práce - Ing. Jiřím Hunkou. Kromě těchto schůzek byl pro komunikaci využíván systém projektového řízení Redmine (popsáno v \ref{infrastructure:redmine}). Další komunikace nebyla jednotně stanovena a probíhala tak v různých komunikačních kanálech.
TODO rozvoj, Slack...

\subsection{Zpětná vazba}

Zpětná vazba probíhá standardně ze strany vedení projektu. Před realizací této práce tak byla veškerá zpětná vazba realizována především na pravidelných týdenních schůzkách, příležitostě také v systému probjektového řízení - Redmine.
Týmy také měly stanoveny své vedoucí, kteří řídili tým interně. Vedoucí týmu na schůzce prezentuje výstupy jeho týmu a popisuje, co který člen realizoval. Týmy se samy interně hodnotí v několika iteracích rozložených do celého semestru.
TODO rozvoj, bodování

\section{Nástroje}

Pro jakékoliv řízení projektu je potřeba využívat určitých nástrojů. Správě infrastruktury v projektovém řízení je v této práci věnována samostatná kapitola \ref{ch:infrastructure}.
